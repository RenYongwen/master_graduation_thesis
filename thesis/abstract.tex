% -*- coding: utf-8 -*-


\begin{zhaiyao}
\begin{spacing}{1.5}
{

本文针对语音语种识别技术的研究现状和发展趋势进行了探讨,介绍了传统的GMM、HMM等模型以及近年来兴起的深度学习模型和自监督预训练模型的应用情况。针对深度学习模型中的ECAPA-TDNN模型进行了详细介绍,分析了该模型在语音信号处理方面的优势和应用前景,包括对变化的语音信号具有更强的鲁棒性,以及对于小数据集的训练具有更好的表现等。在实验部分,本文将ECAPA-TDNN模型应用于语音语种识别任务中,设计了对比实验以验证模型的有效性,包括使用不同的超参数和损失函数等,从而分析得出了影响该模型
性能表现的可能因素。最后,本文将该模型部署到了网页系统以方便用户使用。该研究为语音语种识别技术的进一步发展和实际应用提供了有益的探索和参考。
}
\end{spacing}
\end{zhaiyao}




\begin{guanjianci}
ECAPA-TDNN,语音语种识别,深度学习
\end{guanjianci}



\begin{abstract}
\begin{spacing}{1.5}
This article discusses the research status and development trends of speech language identification technology, including traditional models such as GMM and HMM, as well as recent advances in deep learning models and self-supervised pre-training models. The ECAPA-TDNN model in particular is described in detail, analyzing its advantages and potential applications in speech signal processing, including stronger robustness to changing speech signals and better performance on small datasets. In the experimental section, the ECAPA-TDNN model is applied to speech language identification tasks, and comparative experiments are designed to verify the effectiveness of the model, including the use of different hyperparameters and loss functions, in order to analyze potential factors that affect model performance. Finally, the model is deployed to a web system for ease of use by users. This study provides valuable exploration and reference for the further development and practical application of speech language identification technology.
\end{spacing}
\end{abstract}


\begin{keywords}
ECAPA-TDNN,speech language identification,deep learning
\end{keywords} 
